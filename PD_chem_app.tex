\documentclass[11pt]{article}
% RFP specifically says to use 11 point type and 1 inch margins
\usepackage{graphicx}
\usepackage{epsf,color}
\textwidth=6.5in\oddsidemargin=0in \evensidemargin=0in \topmargin
0pt \advance \topmargin by -\headheight \advance \topmargin by
-\headsep \textheight 9.0in

%\textwidth=6.5in\oddsidemargin=0in \evensidemargin=0in \topmargin
%0pt \advance \topmargin by -\headheight \advance \topmargin by
%-\headsep \textheight 8.9in

\usepackage{amsmath}
\usepackage{graphicx}
\usepackage{dcolumn}
\usepackage{multirow}
\usepackage{wrapfig}
\usepackage[compact]{titlesec}

%\usepackage[plain]{fullpage}
\usepackage{amsfonts}
%\usepackage{lastpage}
%\usepackage{fancyhdr}

\usepackage[version=3]{mhchem} 
% you can use this command to skip chunks of your document
% just put the command around the chunk like this
% \comment{ ...the chunk... }
\newcommand{\comment}[1]{}

%\newcommand{\MarginPar}[1]{\hspace{1sp}\marginpar{\tiny\sffamily\raggedright\hspace{1sp}#1}}
\setlength{\marginparwidth}{0.75in}
\newcommand{\MarginPar}[1]{\marginpar{%
\vskip-\baselineskip %raise the marginpar a bit
\raggedright\tiny\sffamily
\hrule\smallskip{\color{red}#1}\par\smallskip\hrule}}

\definecolor{drkgrn}{rgb}{0.043,0.341,0.2274}
\newcommand{\remrg}[1]{ {\it \color{drkgrn} \{#1 -RG\}}}

%\renewcommand{\baselinestretch}{1.05} % = 1.0 Single space; = 2.0 Double
\renewcommand{\baselinestretch}{1.0} % = 1.0 Single space; = 2.0 Double

%\renewcommand{\refname}{Literature Cited}
%------------------------

%\pagestyle{empty}  % No page numbers
%\textfloatsep 0mm
%\abovecaptionskip 1mm

\begin{document}

%\pagestyle{plain}
%\pagenumbering{roman}
\begin{center}
{\Large{\textbf{Point defect formation kinetics}}}
\end{center}

 
\subsection*{Point defect formation in thin-film photovoltaics:}
A key design need for high efficiency photovoltaic devices based on
newer materials is the ability to control intrinsic point defect
formation to achieve effective n+ and p+ doping. Unlike traditional
(e.g., $Si$) materials where dopant atoms are purposefully introduced,
these newer materials (such as CZTS) are self doped by intrinsic
defects including vacancies, antisite and interstitial defects
\cite{JiangY13}. The creation of such defects during thin film
deposition can be modelled by a set of kinetic rate equations that
incorporate static enthalpy differences obtained from electronic
structure calculations \cite{ref} in a `point-defect chemistry'
paradigm. For example, an abstract metal-sulfide system could be
described by a simplistic system of four reactions expressing an
oxidation reaction, the Schottky disorder and creation of anion
vacancies. Systems of practical interest (e.g., CZTS, CIGS, etc.) are
far more complicated and involve 10s-100s of possible defect formation
relations. 

\begin{eqnarray*}
\ce{ \tfrac{1}{2} S2 -> S_s^x + V_m^{''} + 2 h^+ } \\
\ce{ NULL -> V_m^{''}+ V_s^{$\cdot \cdot$} } \\
\ce{ NULL -> e^' + h^{$\cdot$} } \\
\ce{ S_s^x -> \tfrac{1}{2} S2 + V_s^{$\cdot \cdot$} + 2 e^- }.
\end{eqnarray*}

Where the rate of the $\alpha^{\mathrm{th}}$ reaction is approximated
by the Arrhenius form:
\begin{equation}
  \label{eq:1}
  k_\alpha e^{-(E_a)_\alpha/(RT)}
\end{equation}

As in the combustion example, a hierarchy of simulations with varying
physical realism can be performed, e.g., diffusion of the non-metal
species ($S_2$ above) into the film can be included and may be
rate-limiting There are multiple levels of fidelity that can be
sought to create a computational model based on the above ideas.
\begin{itemize}
\item At the most simplistic, the raw energy differences across each
  reaction can be used to solve for equilibrium concentrations in an
  homogenous (0-D) configuration.
\item Alternatively, a time-dependent solution can be sought. This
  adds complexity to the model and input parameters: it requires kinetic
  rate pre-exponential factors but can also handle time-dependent
  boundary conditions and arrive a quasi-equilibrium states depending
  on the initial conditions.
\item Model fidelity can also be improved by increasing spatial
  dimension; 1-D models admit transport limited solutions, and 2/3-D
  models allow additional interactions with grain and phase
  boundaries. A simple 1-D gradient transport model is to approximates the
  diffusion rate by a Fickian process characterized by $D_j$ for the
  $j^\mathrm{th}$ species. 
\end{itemize}

Even in the 0-D case, for the M-S For the such a system, the ambient
$S_2$ partial pressure is a key parameter that when varied results in
equilibrium states with different doping (performance)
characteristics. Time-dependent histories of the external conditions
during crystal growth can be expected to access quasi-equilibrium
states. For realistic systems, involving 10s to 100s of such kinetic
expressions, many such states with desirable properties may exist.

\begin{figure}[h]
  \centering
  \includegraphics[width=0.8\textwidth]{pd_sketch.pdf}
  \caption{Sketch of film deposition}
  \label{fig:film}
\end{figure}

Direct measurement of the concentration of arbitrary point defects is
difficult and must be inferred from more easily observed
characteristics, leading a model for the observation process $h(x)$
that may be quite complicated and the result of a non-trivial
simulation. Complicating comparison to simulations, although
experiments can be designed to match the assumptions a (necessarily)
incomplete model as closely as possible, this is not necessarily
possible. For instance, a slow deposition rate and carefully
controlled boundary conditions result in an approximately steady 0-D
situation, yet concentration gradients through the film normal to the
surface can still appear from finite-rate transport. 

\remrg{This seems to be an interesting challenge --- how to separate
  incomplete model from uncertain parameters / measurements}

\remrg{The experimental process parameters could be
  thought of as either an uncertain input, albeit not one we're
  trying to constrain, or as an observable. At first glance the latter
  seems more appropriate.}


\begin{table}[h]
  \centering
  \begin{tabular}{l l l l}
    Level in hierarchy & Model inputs & Observables & Processes
    addressed \\
    \hline
    0-D, steady & $(E_a)_\alpha$  & $p_{S_2}(t_0)$ & Equilibrium
    concentrations \\
    & & $T(t_0)$ & \\
    & & $h^+(t_\infty)$ & \\
    & & $e^-(t_\infty)$ & \\
    \hline
    0-D, unsteady & $(E_a)_\alpha$& $p_{S_2}(t)$ & Time dependent
    bcs \\
    & $k_\alpha$ & $T(t)$ & Quasi-equilibrium states \\
    & & $h^+(t_\infty)$ & Dependence on ICs \\
    & & $e^-(t_\infty)$ & \\

    \hline
    1-D, steady & $(E_a)_\alpha$ &  $p_{S_2}(t_0)$ & Equilibrium
    concentrations \\
     & $D_j$ & $T(t_0)$ & Finite rate transport \\
    & & $h^+(t_\infty)$ & \\
    & & $e^-(t_\infty)$ & \\
    \hline
    1-D, unsteady & $(E_a)_\alpha$& $p_{S_2}(t)$ & Time dependent
    bcs \\
    & $k_\alpha$ & $T(t)$ & Quasi-equilibrium states \\
     & $D_j$ &$h^+(t_\infty)$ &  Dependence on ICs \\
     & & $e^-(t_\infty)$ & Finite rate transport \\
     & && Time dependent bcs coupled \\
     & & &    to finite rate transport \\
  \end{tabular}
  \caption{Inputs, observables and physical processes included in
    possible PD simulation hierarchy. An alternate direction would
    incorporate various complexities of reaction networks. Increasing spatial dimension
    further adds additional physics such as interaction with dislocations, grain/phase boundaries}
  \label{tab:pdh}
\end{table}


\bibliographystyle{plain}

\bibliography{pd.bib} 



\end{document}
