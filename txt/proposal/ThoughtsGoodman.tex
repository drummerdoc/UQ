\documentclass[11pt]{article}
% RFP specifically says to use 11 point type and 1 inch margins
\usepackage{graphicx}
\usepackage{epsf,color}
\textwidth=6.5in\oddsidemargin=0in \evensidemargin=0in \topmargin
0pt \advance \topmargin by -\headheight \advance \topmargin by
-\headsep \textheight 9.0in

%\textwidth=6.5in\oddsidemargin=0in \evensidemargin=0in \topmargin
%0pt \advance \topmargin by -\headheight \advance \topmargin by
%-\headsep \textheight 8.9in

\usepackage{amsmath}
\usepackage{graphicx}
\usepackage{dcolumn}
\usepackage{multirow}
\usepackage{wrapfig}
\usepackage[compact]{titlesec}

%\usepackage[plain]{fullpage}
\usepackage{amsfonts}
%\usepackage{lastpage}
%\usepackage{fancyhdr}

\usepackage[version=3]{mhchem} 
% you can use this command to skip chunks of your document
% just put the command around the chunk like this
% \comment{ ...the chunk... }
\newcommand{\comment}[1]{}

%\newcommand{\MarginPar}[1]{\hspace{1sp}\marginpar{\tiny\sffamily\raggedright\hspace{1sp}#1}}
\setlength{\marginparwidth}{0.75in}
\newcommand{\MarginPar}[1]{\marginpar{%
\vskip-\baselineskip %raise the marginpar a bit
\raggedright\tiny\sffamily
\hrule\smallskip{\color{red}#1}\par\smallskip\hrule}}

%\renewcommand{\baselinestretch}{1.05} % = 1.0 Single space; = 2.0 Double
\renewcommand{\baselinestretch}{1.0} % = 1.0 Single space; = 2.0 Double

%\renewcommand{\refname}{Literature Cited}
%------------------------

%\pagestyle{empty}  % No page numbers
%\textfloatsep 0mm
%\abovecaptionskip 1mm

\begin{document}

%\pagestyle{plain}
%\pagenumbering{roman}
\begin{center}
{\Large{\textbf{Project Thooughts}}}

\vskip\baselineskip
{\large{J. Bell, M. Day, J. Goodman, P. Graf, R. Grout, M. Morzfeld, G. Pau}}
\end{center}

This is some text that can be copied and pasted.  
Clearly, I'm the sous chef, not the idea guy.

\section{Monte Carlo Technique}

A central part of this effort will be developments in computational stochastics needed to 
carry out the science.
Bayesian uncertainty quantification is easy to formulate as a problem in posterior sampling.
But some of those sampling problems are impossible given current sampling techniques.
This applies in three related situations that new sampling ideas may address
\begin{enumerate}
\item Particle filtering with some low noise observations
\item Highly ill conditioned posterior distributions with approximate degeneracies in the parameters
\item Problems with a range of length and time scales, such as arise in stochastic PDE
\end{enumerate}
We will use and extend existing software and algorithms to address these issues, using some of the
following.

Implicit sampling is a technique for particle filtering problems in which some low noise observations
highly constrain the random variables to be sampled at each time step.
In these cases, a Gaussian type approximation of the desired distribution allows proposals that
are much closer to the target distribution than a simple Bayesian prior.
One step in this is the optimization problem to find the maximum likelihood point.
We're G-d. (Actually, Chorin and collaborators)

A model has an approximate degeneracy if many different parameter sets are nearly as good at explaining
the data.
For example, if there are multiple reaction pathways, the sum of the reaction rates may be much
better estimated than the individual rates.
In such cases, isotropic sampling algorithms such as single variable heat bath (Gibbs sampler) or
isotropic Metropolis walk will be slow.
We're G-d on this too.

Probability distributions defined by field theories and stochastic PDE have fluctuations on a wide
range of length scales.
There are multi-scale sampling algorithms (Swendsen Wang, Chorin's ``chainless'', Goodman Sokal, etc.)
that address this.
Much work remains to be done to make these more general.
The posterior distributions that arise in the fluid/reaction rate identification problem of combustion
(section 1?) is a perfect test problem to test and develop these ideas further.
The next generation discretization algorithms for stochastic PDE developed by Bell et.al. represent 
new challenges for multi-scale sampling.



\section{Effect of incorrect models}

It is important to use statistical methods that are robust in the inevitable situation where the
exact model is not in the family being fit.
The power law/exponential formulas for reaction rates are only modeling approximations, though they
may be very accurate.
In chaotic systems especially, even small modeling errors may make an accurate global fit impossible.
One approach is to include noise in the dynamics, so that the posterior distribution does not require
the dynamical equations to be satisfied exactly.
We will conduct computational experiments to study this problem, then use the results to 
choose appropriate noise levels for our physical models.






\end{document}
